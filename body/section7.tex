% !TeX root = ../main.tex
% This is section 7.

\section{紧性}

\subsection{紧空间}

    开覆盖, 子覆盖和有限覆盖等概念已经在数学分析1和实分析中详细介绍, 此处不再赘述.

    \begin{Definition}[紧空间]
        设 $ X $ 是拓扑空间, 若 $ X $ 的任意开覆盖都有有限子覆盖, 则称 $ X $ 是\emph{紧空间}. 若 $ A\subset X $ 作为 $ X $ 的子空间是紧空间, 则称 $ A $ 是 $ X $ 的\emph{紧子集}.
    \end{Definition}

    于是由以上定义可知 $ A $ 是 $ X $ 的紧子集当且仅当 $ A $ 在 $ X $ 中的任何开覆盖都有有限子覆盖.

    \begin{Example}
        一些典型的拓扑空间的例子:
        \begin{enumerate}
            \item 有限个元素构成的拓扑空间是紧的.
            \item 平凡拓扑空间是紧的.
            \item 在 $ \R $ 上, $ [a,b] $ 是紧的, 但 $ (a,b) $ 不是紧的. 这因 $ \set{(a,b-(b-a)/n) : n\in\Zi} $ 无有限子覆盖. 同样地, $ \R\simeq(a,b) $ 也不是紧的.
            \item 有限补空间 $ (\R,\ST_F) $ 是紧的, 但可数补空间 $ (\R,\ST_C) $ 不是紧的. 这因 $ \set{(\R\sm\Z)\cup\set{n} : n\in\Z} $ 无有限子覆盖.
        \end{enumerate}
    \end{Example}

    因为开集和闭集是对偶的, 于是可以用闭集来刻画紧性:

    \begin{Definition}[有限交性质]
        设 $ X $ 是拓扑空间, $ \CA\subset 2^X $, 若 $ \CA $ 的任何有限子族都有非空的交, 即 $ \forall\CB\in\fin\CA\,(\bigcap\CB\ne\varnothing) $ , 则称 $ \CA $ 具有\emph{有限交性质}.
    \end{Definition}
    于是对偶地有

    \begin{Proposition}
        $ X $ 是紧空间当且仅当 $ X $ 中具有有限交性质的闭集族都有非空的交.
    \end{Proposition}

    下面做一些闭区间性质的推广:

    \begin{Proposition}
        紧空间的闭子集是紧的, 而 $ T_2 $ 空间的紧子集是闭的.
    \end{Proposition}
    \begin{Proof}
        (1) 设 $ X $ 是紧空间, $ A\subset X $ 是其闭子集, 则 $ X\sm A $ 是开集. 任取 $ A $ 在 $ X $ 中的开覆盖 $ \CC=\set{U_i}_{i\in I} $, 再令 $ \CC'=\CC\cup\set{X\sm A} $. 则 $ \CC' $ 是 $ X $ 的开覆盖. 又因为 $ X $ 是紧的, 故存在 $ \CC_F' $ 是 $ \CC' $ 的有限子覆盖. 取
        \[
            \CC_F=\CC'_F\sm\set{X\sm A}
        \]
        是 $ \CC $ 的有限子覆盖.

        (2) 设 $ X $ 是 $ T_2 $ 空间, $ A\subset X $ 是其紧子集. 则
        \[
            \forall x\in X\sm A\,\forall a\in A\,\exists U_a\in\CN_A(a)\,\exists V_a\in\CN(x)\,(U_a\cap V_a=\varnothing),
        \]
        其中 $ \CN_A(a) $ 表示 $ A $ 中 $ a $ 的邻域系. 因 $ A $ 是紧的, 注意到 $ \set{U_a}_{a\in A} $ 是 $ A $ 的开覆盖, 故存在有限子覆盖 $ \{U_{a_i}\}_{i=1}^k $. 取 $ V=\bigcap_{i=1}^k V_{a_i} $, $ U=\bigcup_{i=1}^k U_{a_i} $, 有 $ A\subset U $ 且 $ V\in\CN(x) $. 故
        \[
            V\cap A\subset U\cap V=\bigcup_{i=1}^k(U\cap V_{a_i})=\varnothing,
        \]
        即 $ x\in U\subset X\sm A $, 即 $ X\sm A $ 是开集. 故 $ A $ 是闭集.\qed
    \end{Proof}

    \begin{Corollary}
        紧 $ T_2 $ 空间上的子集是闭的当且仅当它是紧的.
    \end{Corollary}

    \begin{Proposition}
        紧 $ T_2 $ 空间是 $ T_4 $ 的.
    \end{Proposition}

    \begin{Proposition}
        设 $ A $ 是 $ \R $ 的子集, 则 $ A $ 紧当且仅当 $ A $ 有界闭.
    \end{Proposition}
    \begin{Proof}
        \textsl{必要性}. 因 $ \R $ 是 $ T_2 $ 的, 于是 $ A $ 是紧集可以推出 $ A $ 是闭集. 而 $ A $ 有开覆盖
        \[
            \CC=\set{(-n,n)}_{n\geqslant 1},
        \]
        于是由 $ A $ 的紧性可知 $ \CC $ 存在有限子覆盖, 从而 $ A $ 有界.

        \textsl{充分性}. 设 $ A $ 是 $ \R $ 中的有界闭集, 则存在 $ M>0 $ 使得 $ A\subset[-M,M] $, 且 $ A $ 是 $ [-M,M] $ 的闭子集. 而 $ [-M,M] $ 是紧的 $ T_2 $ 空间, 于是 $ A $ 是紧的.\qed
    \end{Proof}

    由空间的紧性可得命题 \ref{prop:正则空间的刻画} 的一个推论:

    \begin{Corollary}
        紧正则空间一定是正规的.
    \end{Corollary}

    但反之不成立, 若取
    \[
        X=\set{1,2,3},\quad \ST=\set{\varnothing,\set{1},\set{2},\set{1,2},X}
    \]
    则 $ (X,\ST) $ 正规而且是紧空间, 但 $ (X,\ST) $ 不是正则空间.

    \begin{Proposition}
        紧集在连续映射下有以下性质:
        \begin{enumerate}
            \item 紧集的连续像是紧的;
            \item 设 $ f : X\to Y $ 是连续映射, $ X $ 是紧的, $ Y $ 是 $ T_2 $ 空间, 则 $ f $ 是闭映射. 特别地, 当 $ f $ 是双射时它必是同胚.
            \item (\emph{极值定理}) 设 $ f : X\to\R $ 是连续函数, 其中 $ X $ 是紧的. 则 $ f $ 有界, 且在 $ X $ 上取到最大值和最小值.
        \end{enumerate}
    \end{Proposition}
    \begin{Proof}
        (1) 设 $ X $ 是紧集, $ f : X\to Y $ 连续, 任取 $ f(X) $ 的开覆盖 $ \CC=\set{U_i}_{i\in I} $, 则 $ \CC_X=\set{f^{-1}(U_i)}_{i\in I} $ 是 $ X $ 的开覆盖. 由 $ X $ 的紧性可知 $ \CC_X $ 存在有限子覆盖 $ \CC'_X=\set{f^{-1}(U_j)}_{j\in J} $, 于是 $ \CC'=\set{U_j}_{j\in J} $ 是 $ f(X) $ 的有限子覆盖. 于是 $ f(X) $ 也是紧的.

        (2) 任取 $ X $ 的闭子集 $ F $, 则 $ X $ 紧可以推出 $ F $ 紧. 由(1)知 $ f(F) $ 也是紧的, 但 $ Y $ 是 $ T_2 $ 空间, 故 $ f(F) $ 是 $ Y $ 的闭子集, 从而 $ f $ 是闭映射.

        (3) 由(1)可知 $ f(X) $ 是 $ \R $ 上的紧集, 从而是有界闭集, 于是 $ m=\inf f(X) $ 和 $ M=\sup f(X) $ 都是 $ f(X) $ 中的元素, 即 $ f $ 可以在 $ X $ 上取到最大值和最小值.\qed
    \end{Proof}

\subsection{积空间上的紧性}

    \begin{Proposition}[管状邻域引理]\label{prop:管状邻域引理}
        设 $ (X,\ST_X),\ (Y,\ST_Y) $ 是拓扑空间, $ Y $ 是紧空间, $ N $ 是 $ X\times Y $ 的开子集, $ x_0\in X $ 且 $ \set{x_0}\times Y\subset N $, 则存在包含 $ x_0 $ 的开子集 $ U $ 使得 $ U\times Y\subset N $.
    \end{Proposition}
    \begin{Proof}
        取 $ \set{U\times V : U\in\ST_X, V\in\ST_Y} $ 是 $ X\times Y $ 的基, 则基中包含在 $ N $ 内且与 $ \set{x_0}\times Y $ 香蕉的所有开机构成 $ \set{x_0}\times Y $ 的开覆盖, 记其为 $ \CC $.

        因 $ \set{x_0}\times Y\simeq Y $ 是紧的, 于是存在 $ \CC $ 的有限子覆盖 $ \CC'=\set{U_i\times V_i : i=1,2,\dots,k} $. 从而 $ \{V_i\}_{i=1}^k $ 是 $ V $ 的开覆盖. 令 $ U=\bigcap_{i=1}^n U_i $, 则 $ x_0\in U $, $ U $ 是开集, 且
        \[
            U\times Y\subset\bigcup_{i=1}^nU\times V_i\subset\bigcup_{i=1}^nU_i\times V_i\subset N,
        \]
        得证.\qed
    \end{Proof}

    \begin{Theorem}
        紧空间的有限积空间是紧的.
    \end{Theorem}
    \begin{Proof}
        只需证明两个空间的情形, 有限情形可由归纳法证明. 为此设 $ (X,\ST_X) $ 和 $ (Y,\ST_Y) $ 都是紧空间, 取 $ \CC $ 是 $ X\times Y $ 的开覆盖. 对任意 $ x\in X $, 注意到 $ \set{x}\times Y\simeq Y $ 是紧的, 于是存在 $ \CC $ 的 的有限子覆盖 $ \{C_i\}_{i=1}^m $ 使得
        \[
            \set{x}\times Y\subset\bigcup_{i=1}^mC_i
        \]
        由命题 \ref{prop:管状邻域引理} 可知存在包含 $ x $ 的开子集 $ U_x $ 使得 $ U_x\times Y\subset\bigcup_{i=1}^mC_i $, 而 $ \set{U_x}_{x\in X} $ 是 $ X $ 的开覆盖, 由 $ X $ 的紧性可知存在有限子覆盖 $ \{U_j\}_{j=1}^n $ 使得 $ X\subset\bigcup_{j=1}^nU_j $. 故对 $ x_j\in X $ 有
        \[
            U_j\times Y\subset\bigcup_{i=1}^mC_{ij},
        \]
        从而
        \[
            X\times Y\subset\bigcup_{j=1}^nU_j\times Y\subset\bigcup_{j=1}^n\bigcup_{i=1}^mC_{ij},
        \]
        而右侧是 $ \CC $ 的有限子覆盖.\qed
    \end{Proof}

    由上述有限积空间的结论可知 $ \R^n $ 的子集是紧的当且仅当它是有界闭的, 而单位圆盘 $ \D $, $ n $--环面 $ \T^n $ 和球面 $ \S^n $ 都是紧的.

    更一般地, 有以下定理成立, 它的证明依赖 Zorn 引理, 只需例行地证明即可:

    \begin{Theorem}[Tychonoff]
        设 $ \set{X_i}_{i\in I} $ 是一族紧空间, $ I $ 是任意指标集, 则 $ \prod_{i\in I}X_i $ 也是紧的.
    \end{Theorem}

\subsection{可数紧性\ 序列紧性\ 聚点紧性}

    \begin{Definition}[可数紧, 序列紧, 聚点紧]
        设 $ (X,\ST) $ 是一个拓扑空间,
        \begin{enumerate}
            \item 若 $ X $ 的任意可数开覆盖都存在有限子覆盖, 则称 $ X $ 是\emph{可数紧}的.
            \item 若 $ X $ 的任意开覆盖都存在可数子覆盖, 则称 $ X $ 是 \emph{Lindel\"of}的.
            \item 若 $ X $ 的任意无限子集都有聚点, 则称 $ X $ 是\emph{聚点紧}的.
            \item 若 $ X $ 的任意序列都有收敛子列, 则称 $ X $ 是\emph{序列紧}的.
        \end{enumerate}
    \end{Definition}

    由上述定义可知紧空间一定是可数紧的, 且可数紧的 Lindel\"of 空间一定是紧的.

    \begin{Proposition}[可数紧空间的刻画]
        $ X $ 是可数紧空间当且仅当 $ X $ 中任意单调递减的非空闭集序列都有非空的交.
    \end{Proposition}
    \begin{Proof}
        \textsl{必要性}. 设 $ (A_n)_{n\geqslant 1} $ 是单调递减的非空闭集列, 若 $ \bigcap_{n\geqslant 1}A_n=\varnothing $, 由 $ \left(\bigcap_{n\geqslant 1}A_n\right)^c=X $ 可知 $ \{A_n^c\}_{n\geqslant 1} $ 是 $ X $ 的一个可数开覆盖, 因 $ X $ 可数紧, 故它存在有限子覆盖 $ \{A_{n_i}^c\}_{i=1}^k $, 而
        \[
            \varnothing=X^c=\left(\bigcup_{i=1}^kA_{n_i}^c\right)=\bigcap_{i=1}^kA_{n_i}=A_{\max n_i}\ne\varnothing,
        \]
        矛盾.

        \textsl{充分性}. 用反证法, 若 $ X $ 不是可数紧的, 则存在 $ X $ 的可数开覆盖 $ \CC=\{G_n\}_{n\geqslant 1} $ 使得 $ \CC $ 不存在有限子覆盖, 则 $ \forall n\in\Zi $, 取 $ V_n=\bigcup_{i=1}^nG_i $ 后有 $ \{V_n\}_{n\geqslant 1} $ 是 $ X $ 的开覆盖但不存在有限子覆盖, 且 $ (V_n^c)_{n\geqslant 1} $ 是非空闭集列, 它满足 $ \bigcup_{n\geqslant 1}V_n^c=\varnothing $, 矛盾.\qed
    \end{Proof}

    借助以上对可数紧空间的刻画, 可以进一步讨论可数紧与其他紧性的关系:

    \begin{Proposition}
        可数紧空间一定是聚点紧的, $ T_1 $ 的聚点紧空间是可数紧的.
    \end{Proposition}
    \begin{Proof}
        (1) 若 $ X $ 不是聚点紧的, 取 $ X $ 的一个无聚点可数子集 $ A=\set{x_n}_{n\geqslant 1} $. 由 $ A $ 无聚点可知 $ A $ 是闭集, 即 $ X\sm A $ 是开集. 而对任意 $ n\in\Zi $, 由 $ x_n\notin d(A) $ 可知 $ \exists U_n\in\CN(x_n) $ 使得 $ U_n\cap(A\sm\set{x_n})=\varnothing $, 因此
        \[
            \CC=\set{X\sm A, U_n : n\geqslant 1}
        \]
        是 $ X $ 的可数子覆盖, 但它不存在有限子覆盖, 这因若存在 $ i_1,\dots,i_n $ 有 $ A=A\cap\left(\bigcup_{k=1}^nU_{i_k}\right)=\set{a_{i_1},\dots,a_{i_n}} $ 是有限集, 矛盾.

        (2) 用反证法, 设 $ X $ 中存在单调递减的非空闭集列 $ (F_n)_{n\geqslant 1} $ 满足 $ \bigcap_{n\geqslant 1}=\varnothing $, 任取 $ x_n\in F_n $, 并取 $ A=\set{x_n}_{n\geqslant 1} $.

        若 $ A $ 是有限集, 则存在 $ x\in A $ 和一列单调递增的正整数列 $ (n_i)_{i\geqslant 1} $ 使得 $ x=x_{n_i},\ \forall i=1,2,\dots $, 故
        \[
            \forall k\in\Zi\,(x=x_{n_k}\in F_{n_k}\subset F_k),
        \]
        即 $ x\in\bigcap_{k\geqslant 1}F_k $, 矛盾.

        若 $ A $ 是无限集, 由于 $ X $ 聚点紧, 故存在 $ y\in d(A)\ne\varnothing $. 又因为 $ X $ 是 $ T_1 $ 的, 故所有有限子集都是闭集. 对任意 $ n\in\Zi $, 取 $ A_n=\set{x_k}_{k\geqslant n} $ 后有 $ y\in d(A_n)\ne\varnothing $. 由 $ A_n $ 的构造和 $ F_n $ 单调递减可知 $ \forall n\in\Zi\,(A_n\subset F_n) $. 又由 $ F_n $ 是闭集可知
        \[
            \forall n\in\Zi\,(y\in d(A_n)\subset\baro{A_n}\subset\baro{F_n}=F_n),
        \]
        即 $ y\in\bigcap_{n\geqslant 1}F_n $, 矛盾.\qed
    \end{Proof}

    \begin{Proposition}
        序列紧空间一定是可数紧的, $ C_1 $ 的可数紧空间一定是序列紧的.
    \end{Proposition}
    \begin{Proof}
        (1) 设 $ X $ 是序列紧的, $ (F_n)_{n\geqslant 1}\subset X $ 是一列单调递减的非空闭集, 任取 $ x_n\in F_n $. 由 $ X $ 序列紧可知 $ (x_n)_{n\geqslant 1} $ 由收敛子列 $ (x_{n_k})_{k\geqslant 1} $, 于是
        \[
            \forall k\in\Zi\,(x\in\baro{F_{n_k}}=F_{n_k}\subset F_k),
        \]
        即 $ x\in\bigcap_{k\geqslant 1}F_k $, 则 $ X $ 可数紧.

        (2) 任取 $ X $ 中的序列 $ (x_n)_{n\geqslant 1} $, 取 $ A_n=\set{x_i}_{i\geqslant n} $, 则 $ (\baro{A_n})_{n\geqslant 1} $ 是单调递减的非空闭集序列. 因 $ X $ 可数紧, 于是存在 $ x\in\bigcap_{n\geqslant 1}\baro{A_n} $. 又因为 $ X $ 是 $ C_1 $ 的, 故 $ x $ 是可数邻域基 $ (U_n)_{n\geqslant 1} $, 并不妨设 $ (U_n)_{n\geqslant 1} $ 也是单调递减的. 于是
        \[
            \begin{aligned}
                & x\in\baro{A_1}\,\exists n_1\in\Zi\,(x_{n_1}\in U_1\cap A_1)\\
                & x\in\baro{A_{n_1+1}}\,\exists n_2>n_1\,(x_{n_2}\in U_2\cap A_{n_1+1}),
            \end{aligned}
        \]
        一直进行下去可以得到一列 $ (x_{n_k})_{k\geqslant 1} $ 收敛到 $ x $, 即 $ X $ 序列紧. 
    \end{Proof}

    下面的图表给出了各类紧性的关系:
    \begin{center}
        \begin{tikzpicture}
            \node (Count) at (0,0) {可数紧};
            \node (Cpact) at (-2.5,1.45) {紧};
            \node (Seq) at (2.5,1.45) {序列紧};
            \node (Point) at (0,-2.89) {聚点紧};
            \draw[->] (Count) to[bend right=30] node[right]{\small{Lindel\"of}} (Cpact);
            \draw[->] (Cpact) to[bend right=30] (Count);
            \draw[->] (Count) to[bend right=30] node[right]{$ C_1 $} (Seq);
            \draw[->] (Seq) to[bend right=30] (Count);
            \draw[->] (Count) to[bend right=30] (Point);
            \draw[->] (Point) to[bend right=30] node[right]{$ T_1 $} (Count);
        \end{tikzpicture}
    \end{center}

\subsection{度量空间上的紧性}

    \begin{Theorem}[Lebesgue 覆盖引理]
        设 $ \CC $ 是进度量空间 $ X $ 的开覆盖, 则
        \[
            \exists\delta >0\,\forall x\in X\,\exists C\in\CC\,(B(x,\delta)\subset C),
        \]
        具有这样性质的 $ \delta $ 称作覆盖 $ \CC $ 的 \emph{Lebesgue 数}.
    \end{Theorem}
    \begin{Proof}
        因 $ X $ 是紧的, 故覆盖 $ \CC $ 存在有限子覆盖 $ \CC'=\{C_i\}_{i=1}^n $, 在其上定义连续映射
        \[
            f : X\to\R,\qquad x\mapsto\max_{1\leqslant i\leqslant n}d(x,X\sm C_i),
        \]
        因 $ \forall x\in X\,\exists C_i\,(x\in C_i) $, 故 $ x\notin X\sm C_i $. 而由 $ X\sm C_i $ 是闭的, 故 $ d(x,X\sm C_i)>0 $, 从而 $ f(x)>0 $. 由极值定理取 $ m=\min f(X) $, 再取 $ \delta $ 满足 $ 0<\delta<m $ 即可.\qed
    \end{Proof}

    紧性是极值定理和下面的一致连续定理得以成立的拓扑性质.

    \begin{Definition}[一致连续映射]
        设 $ (X,d) $ 和 $ (Y,d') $ 都是度量空间, $ f : X\to Y $ 是映射, 若
        \[
            \forall\varepsilon>0\,\exists\delta>0\,\forall x_1,x_2\in X\,(d(x_1,x_2)<\delta\Rightarrow d'(f(x_1),f(x_2))<\varepsilon),
        \]
        则称 $ f $ 是\emph{一致连续}的.
    \end{Definition}

    \begin{Proposition}
        紧度量空间到度量空间的连续映射一定一致连续.
    \end{Proposition}
    \begin{Proof}
        设 $ f : X\to Y $ 连续而 $ (X,d) $ 是紧空间, 则对任意 $ \varepsilon>0 $, 有
        \[
            \CC=\set{f^{-1}\left(B\left(y,\frac{\varepsilon}{2}\right)\right) : y\in Y}
        \]
        是 $ X $ 的一个开覆盖. 由 Lebesgue 覆盖引理可知存在 Lebesgue 数 $ \delta $, 于是
        \[
            \forall x_1,x_2\in X\,\left( d(x_1,x_2)<\delta\Rightarrow\exists y_0\in Y\,\left( x_1,x_2\in f^{-1}\left( B\left( y_0,\frac{\varepsilon}{2} \right) \right) \right) \right)
        \]
        即 $ f(x_1),f(x_2)\in B(y_0,\varepsilon/2) $, 故 $ d'(f(x_1),f(x_2))<\varepsilon $, 即 $ f $ 一致连续.\qed
    \end{Proof}

    \begin{Theorem}[紧等价]
        设 $ X $ 是度量空间, 则下列各条件等价:
        \begin{enumerate}
            \item $ X $ 是紧的;
            \item $ X $ 是序列紧的;
            \item $ X $ 是可数紧的;
            \item $ X $ 是聚点紧的.
        \end{enumerate}
    \end{Theorem}

\subsection{局部紧化和单点紧化}

    尽管紧性有许多非常良好的性质, 但遗憾的是常见的 Euclid 空间 $ \R^n $ 却不是紧的. 我们希望有一类比紧性稍弱一些的性质, 能将更多的空间包括进来.

    \begin{Definition}[局部紧]
        设 $ X $ 是拓扑空间, $ x\in X $ 是其中任意一点. 若存在 $ x $ 的紧邻域, 则称 $ X $ 在 $ x $ 处\emph{局部紧}. 若 $ X $ 在每一点都局部紧, 则称 $ X $ 是\emph{局部紧空间}.
    \end{Definition}

    \begin{Example}
        局部紧空间的确包含了许多常用的空间, 下面是一些例子:
        \begin{enumerate}
            \item 紧空间一定是局部紧的;
            \item Euclid 空间 $ \R^n $ 是局部紧空间, 但它不是紧的;
            \item 无限集上取离散拓扑得到的空间是局部紧空间, 但它不是紧的;
            \item $ \Q^n $ 不是局部紧空间;
            \item 拓扑正弦曲线不是局部紧空间, 这因它在 $ \mathbf{0} $ 处没有紧邻域.
        \end{enumerate}
    \end{Example}

    \begin{Proposition}
        局部紧是一种拓扑性质, 不难证明:
        \begin{enumerate}
            \item 局部紧空间的闭子集是局部紧的;
            \item 局部紧空间在连续开映射下的像是局部紧的.
        \end{enumerate}
    \end{Proposition}

    \begin{Proposition}
        紧 $ T_2 $ 空间的开子集是局部紧的.
    \end{Proposition}
    \begin{Proof}
        设 $ A $ 是紧 $ T_2 $ 空间 $ X $ 的开子集, $ x\in A $ 是其中任意一点. 因为紧 $ T_2 $ 空间是 $ T_4 $ 的, 故存在开集 $ U $ 使得
        \[
            x\in U\subset\bar{U}\subset A,
        \]
        $ \bar{U} $ 是 $ x $ 在 $ A $ 中的一个邻域, 而它作为紧空间的闭子集也是紧的, 于是 $ A $ 是局部紧的.\qed
    \end{Proof}

    局部紧的意义在于它可以将一个非紧空间嵌入到一个紧空间中, 成为一个紧空间的子空间. 这样的把戏在复分析中已经出现过, 即引入无穷远点来使得扩充复平面变成一个紧空间:

    \begin{Theorem}[单点紧化]
        设 $ (X,\ST) $ 是一个拓扑空间, $ \infty $ 是一个不同于 $ X $ 中任一点的另外一点(这里 $ \infty $ 只是一个记号, 不表示无穷大), 令 $ \Star{X}=X\cup\set{\infty} $, $ \Star{\ST}=\ST\cup\set{\Star{X}\sm C : C\subset X\,\text{是紧的闭子集}} $. 则 $ \Star{\ST} $ 是 $ \Star{X} $ 的一个拓扑. 称 $ (\Star{X},\Star{\ST}) $ 是 $ (X,\ST) $ 的\emph{单点紧化}.
        \begin{enumerate}
            \item $ (X,\ST) $ 是 $ (\Star{X},\Star{\ST}) $ 的子空间, 且 $ X $ 非紧空间时 $ X $ 在 $ \Star{X} $ 中稠密;
            \item $ (\Star{X},\Star{\ST}) $ 是紧空间;
            \item $ (\Star{X},\Star{\ST}) $ 是紧 $ T_2 $ 空间当且仅当 $ (X,\ST) $ 是局部紧 $ T_2 $ 空间.
        \end{enumerate}
    \end{Theorem}

    于是扩充复平面 $ \Star{\C}\simeq\S^2 $ 无非是 $ \C\simeq\R^2 $ 的单点紧化, 更一般地有 $ \S^n $ 是 $ \R^n $ 的单点紧化.