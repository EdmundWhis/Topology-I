% !TeX root = ../main.tex
% This is section 5.

\section{可数性}

\subsection{可数性公理}

    \begin{Definition}[可数性公理]
        设 $ X $ 是一个拓扑空间:
        \begin{enumerate}
            \item 若 $ X $ 中每一点都存在可数的邻域基, 则称 $ X $ 满足\emph{第一可数性公理}, 或 $ C_1 $\emph{公理};
            \item 若 $ X $ 有一个可数的拓扑基, 则称 $ X $ 满足\emph{第二可数性公理}, 或 $ C_2 $\emph{公理}.
        \end{enumerate}
        满足 $ C_1 $ 公理的空间称作 $ C_1 $\emph{空间}, 满足 $ C_2 $ 公理的空间称作 $ C_2 $\emph{空间}.
    \end{Definition}

    \begin{Example}\label{ex:可数性例子}
        在分析学中使用的各类空间通常是满足第一可数性公理的:
        \begin{enumerate}
            \item 度量空间是 $ C_1 $ 的, 这因 $ \forall x\in X $, 都有 $ \CB(x)=\set{B(x,1/n) : n\in\Zi} $ 是 $ x $ 的一个可数邻域基.
            \item Euclid 空间 $ \R^n $ 是 $ C_2 $ 的, 这因 $ \CB=\set{B(x,1/m) : x\in\Q^n, m\in\Zi} $ 是它的一个可数拓扑基.
            \item 离散拓扑空间 $ (X,\ST_D) $ 是 $ C_1 $ 的, 这因 $ \set{x} $ 就是 $ x\in X $ 的邻域基. 但不可数的离散拓扑空间不是 $ C_2 $ 的, 因它的任意拓扑基 $ \CB $ 总是包含 $ \set{\set{x} : x\in X} $, 但其基数大于 $ \aleph_0 $.
        \end{enumerate}
        但赋予某些在分析学中并不好用的拓扑时, 通常的 Euclid 空间也可以不满足可数性公理:
        \begin{enumerate}[start=4]
            \item 有限补拓扑空间 $ (\R^2,\ST_F) $ 不是 $ C_1 $ 的, 否则 $ \mathbf{0}\in\R^2 $ 有一可数的邻域基 $ \set{U_n}_{n\geqslant 1} $, 并不妨假设 $ U_i $ 均为开集, 则 $ F_n=\R^2\sm U_n $ 均有限, 此时
            \[
                \exists x\ne\mathbf{0}\,\left( x\in\bigcap_{n\geqslant 1}U_n=R^2\sm\bigcup_{n\geqslant 1}F_n \right),
            \]
            即对任意 $ n\in\Zi $, 总有 $ U_n $ 不包含于 $ \R^2\sm\set{x} $, 但 $ \R^2\sm\set{x} $ 是 $ \mathbf{0} $ 的一个邻域, 矛盾.
        \end{enumerate}
    \end{Example}

    分析学中使用的各类空间往往 $ C_1 $ 的原因是在 $ C_1 $ 空间上可以用序列刻画连续:

    \begin{Proposition}
        设 $ X $ 是满足 $ C_1 $ 公理的拓扑空间, 则:
        \begin{enumerate}
            \item 对任意 $ x\in X $, 存在 $ x $ 的可数邻域基 $ \set{V_n}_{n\geqslant 1} $ 使得 $ \forall n\in\Zi(V_{n+1}\subset V_n) $;
            \item 设 $ A\subset X $, 则 $ x\in\bar{A} $ 当且仅当 $ \exists(x_n)_{n\geqslant 1}\subset A\,(\lim_{n\to\infty}x_n=x) $;
            \item 设 $ Y $ 是一拓扑空间, $ x\in X $, 则 $ f : X\to Y $ 在 $ x $ 除连续当且仅当 $ \forall (x_n)_{n\geqslant 1}\subset X\,(\lim_{n\to\infty}x_n=x\Rightarrow\lim_{n\to\infty}f(x_n)=f(x)) $, 此即 Heine 定理.
        \end{enumerate}
    \end{Proposition}

    \begin{Example}[非 $ C_1 $ 空间]
        任何含有不可数个元素的可数补拓扑空间都不是 $ C_1 $ 的.
    \end{Example}
    \begin{Proof}
        设 $ X $ 是一不可数集, $ \ST $ 是 $ X $ 上的可数补拓扑, 则对 $ x\in X $, 若 $ x $ 有可数的邻域基 $ \set{U_n}_{n\geqslant 1} $, 存在一列 $ \set{V_n}_{n\geqslant 1}\subset\ST $ 使得 $ V_n\subset U_n $ 且 $ x\in V_n $. 于是对任意的 $ n\in\Zi $ 都有 $ X\sm V_n $ 可数, 且
        \[
            X\sm\bigcap_{n\geqslant 1}U_n\subset X\sm\bigcap_{n\geqslant 1}V_n=\bigcup_{n\geqslant 1}(X\sm V_n)
        \]
        仍是可数集, 于是存在 $ y\in\bigcup_{n\geqslant 1} $ 使得 $ y\ne x $, 此时 $ X\sm\set{y} $ 仍是 $ x $ 的一个邻域, 于是存在 $ n_0\in\Zi $ 使得 $ V_{n_0}\subset X\sm\set{y} $, 矛盾. 于是 $ X $ 不是 $ C_1 $ 的.\qed
    \end{Proof}

\subsection{有关可数性的若干性质}

    \begin{Proposition}
        以下 $ i=1,2 $ :
        \begin{enumerate}
            \item $ C_i $ 空间的子空间仍然是 $ C_i $ 的;
            \item $ C_i $ 空间的有限积空间仍然是 $ C_i $的.
        \end{enumerate}
    \end{Proposition}
    \begin{Proof}
        (1) 是显然成立的.

        (2) 不妨设 $ \CB_k $ 是 $ X_k $ 的可数拓扑基, 则 $ \prod_{k=1}^n\CB_k $ 是 $ \prod_{k=1}^nX_k $ 的可数拓扑基, 在 $ i=1 $ 的情形只需替换成可数邻域基同样可证.\qed
    \end{Proof}

    \begin{Proposition}\label{prop:满连续开映射}
        设 $ f : X\to Y $ 是一个满的连续开映射, 若 $ X $ 是 $ C_i $ 的, 则 $ f(X) $ 也是 $ C_i $ 的, 这里 $ i=1,2 $.
    \end{Proposition}
    \begin{Proof}
        不妨 $ i=2 $, 设 $ \CB $ 是 $ X $ 的一组可数拓扑基. 因 $ f $ 是开映射, 于是 $ \tilde{\CB}=f(\CB) $ 是 $ Y $ 中的可数开集族. 若 $ U\in\ST_Y $, 由 $ f $ 的连续性可知 $ f^{-1}(U)\in\ST_X $, 于是 $ \exists B\in\CB\,(B\subset f^{-1}(U)) $. 又因为 $ f $ 是满射, 于是
        \[
            f(B)\subset f(f^{-1}(U))=U,
        \]
        故 $ \tilde{\CB} $ 是 $ Y $ 的可数拓扑基, $ Y=f(X) $ 是 $ C_2 $ 的.

        对 $ i=1 $ 的情形, 只需要将拓扑基替换成邻域基, 拓扑替换成邻域系即可.\qed
    \end{Proof}

    \begin{Proposition}
        $ C_2 $ 空间一定是 $ C_1 $ 的.
    \end{Proposition}
    \begin{Proof}
        若 $ \CB $ 是 $ X $ 的一个可数拓扑基, 则对任意 $ x\in X $, 有 $ \CB(x)=\{B\in\CB : x\in B\} $ 是 $ x $ 的一个可数邻域基.\qed
    \end{Proof}

    这一命题的逆命题不真, 例 \ref{ex:可数性例子} 中的(3)提供了一个 $ C_1 $ 但非 $ C_2 $ 的例子. 下面的例子说明了命题 \ref{prop:满连续开映射} 中的开映射是必要的:

    \begin{Example}
        $ C_2 $ 空间的满连续闭映射像未必是 $ C_1 $ 的: 在 $ \R $ 上取通常拓扑, 它成为一个 $ C_2 $ 空间. 商映射 $ \R\twoheadrightarrow\R/\Z $ 是一个满连续闭映射, 但商空间 $ \R/\Z $ 不是 $ C_1 $ 的. 为此不妨记其``坍缩点''为 $ c $, 需要证明它没有可数的邻域基. 注意到 $ c $ 的任何邻域 $ U $, 都存在一列正数 $ \set{\delta_n}_{n\in\Z} $ 使得 $ (n-\delta_n,n)\cup(n,n+\delta_n)\subset U $ 对任意 $ n\in\Z $ 成立. 如果其中有可数的邻域基 $ \set{U_i}_{i\in\Z} $, 则对任意的 $ i\in\Z $, 我们选取一列上面的正数 $ \set{\delta_{i,n}}_{n\in\Z} $, 从而考虑
        \[
            V=\set{c}\cup\bigcup_{n\in\Z}((n-\delta_{n,n}/2,n)\cup(n,n+\delta_{n,n}/2))
        \]
        这是一个 $ c $ 的开邻域, 但每一个 $ U_i $ 都不是 $ V $ 的子集, 这因 $ i+\delta_{i,i}/2 $ 落在 $ U_i $ 中却不落在 $ V $ 中.
    \end{Example}

\subsection{$ C_2 $空间和可分性}

    \begin{Definition}[稠密, 可分空间]
        设 $ X $ 是以拓扑空间, $ A\subset X $; 若 $ \bar{A}=X $, 则称 $ A $ 是 $ X $ 的一个\emph{稠密子集}, 或称 $ A $ 在 $ X $ 中\emph{稠密}. 若 $ X $ 存在可数的稠密子集, 则称 $ X $ 是\emph{可分空间}.
    \end{Definition}

    以下命题在泛函分析中已经提到并证明过, 此处仅列举如下:

    \begin{Proposition}
        $ C_2 $ 空间都是可分的.
    \end{Proposition}
    
    \begin{Proposition}
        $ C_2 $ 空间的每个子空间都是可分的.
    \end{Proposition}

    \begin{Proposition}
        可分的度量空间一定是 $ C_2 $ 的.
    \end{Proposition}

    需要注意的是可分空间的子空间未必仍然可分, 为此考虑以下的例子:

    \begin{Example}
        设 $ X $ 是任意的拓扑空间, 在其中引入不同于 $ X $ 中的任意一点 $ \infty $ 后记作 $ \Star{X} $, 在 $ \Star{X} $ 上保留 $ X $ 上的开集, 并将所有包含 $ \infty $ 的集合都取成开集后 $ \set{\infty} $ 是一个稠密子集, 这时只需要取一个不可分空间 $ X $, 就有 $ \Star{X} $ 可分而其子空间 $ X $ 不可分. 
    \end{Example}

    下面的图表给出了可数性和度量空间、\, Euclid空间之间的关系:
    \begin{center}
        \begin{tikzpicture}
            \node (C2) at (0,0) {$ C_2 $};
            \node (C1) at (-1.2,0) {$ C_1 $};
            \node (D) at (-3,0) {度量空间};
            \node (R) at (0,1.2) {$ \R^n $};
            \node (F) at (3,0) {可分};
            \draw[->] (C2) -- (C1);
            \draw[->] (C2) -- (F);
            \draw[->] (R) -- (D);
            \draw[->] (R) -- (C2);
            \draw[->] (D) -- (C1);
            \draw[->] (D) to[bend right=49] node[below]{+可分} (C2);
            \draw[->] (F) to[bend left=49] node[below]{+度量} (C2);
        \end{tikzpicture}
    \end{center}    